\documentclass[
  paper=a3,
  fontsize=20pt,
  parskip=half,
  twocolumn,
  DIV=16,
]{scrartcl}


\usepackage{fixltx2e}

\usepackage{polyglossia}
\setmainlanguage{german}

% Define Colors:
\usepackage{xcolor}
\colorlet{initialcolor}{red!70!black}
\colorlet{titlecolor}{red!70!black}

\usepackage{fontspec}
\setmainfont{Tex Gyre Chorus}
\addtokomafont{disposition}{\rmfamily}

\usepackage{yfonts}
\usepackage{microtype}

\setlength{\columnsep}{70pt}

% \newfontfamily{\initials}{EB Garamond Initials}
\newfontfamily{\initials}{Tex Gyre Chorus}
\newfontfamily{\calligraphic}{Tex Gyre Chorus}
\newfontfamily{\serif}{EB Garamond}

\usepackage{lettrine}
% \renewcommand{LettrineTextFont}{}

\usepackage{xparse}

% \NewDocumentCommand{\initial}{m m o o}<++>

\newcommand{\initial}[3][initialcolor]{%
  \lettrine[
    lhang=0.5,
    lines=2,
    lraise=0.6,
    depth=-1
  ]{\color{#1}\initials#2}{#3}
}

\newcommand{\initialD}[2][initialcolor]{%
  \lettrine[
    lhang=0.5,
    lines=2,
    findent=1pt,
    lraise=0.6,
    depth=-1
  ]{\color{#1}\initials D}{#2}
}

\newcommand{\initialW}[2][initialcolor]{%
  \lettrine[
    lhang=0.5,
    lines=2,
    findent=-2pt,
    lraise=0.6,
    depth=-1
  ]{\color{#1}\initials W}{#2}
}

\newcommand{\initialI}[2][initialcolor]{%
  \lettrine[
    lhang=0.5,
    lines=2,
    findent=-2pt,
    lraise=0.6,
    depth=-1
  ]{\color{#1}\initials I}{#2}
}

\newcommand{\initialG}[2][initialcolor]{%
  \lettrine[
    lhang=0.5,
    lines=2,
    findent=-1pt,
    lraise=0.7,
    depth=-1
  ]{\color{#1}\initials G}{#2}
}


\newcommand{\initialEs}[1][initialcolor]{%
  \vspace{-0.2em}
  \lettrine[
    lhang=0.5,
    lines=2,
    findent=-0.4em,
    lraise=0.6,
    depth=-1
  ]{\color{#1}\initials E}{\color{#1}s}\hspace{0.1em}
}

\usepackage{anyfontsize}
\usepackage{calc}

\newcommand{\poemwitdh}{\widthof{seinem Rauchfleisch und Mehl nicht schaden}}

\pagestyle{empty}

\newcommand{\glaubenhoffenlieben}{%
  \centerline{Dieser Brief ist geschrieben aus}
  \centerline{\serif \textbf{\textsc{Glauben Hoffen Lieben}}}%
}

\begin{document}
\twocolumn[%
  \centering\textswab{\fontsize{114}{120}\selectfont\color{titlecolor} Schutzbrief}
  \Large
  \glaubenhoffenlieben
  \vspace{1cm}
]

\flushbottom

\initialW{er} ihn in seinem Hause hält, \\
dem schlögt kein Hagel ins Weizenfeld, \\
seine Kirschen sind sicher vor Spatzen, \\
kein Wasserrohr darf ihm platzen, \\
kein böses Maul ihn verklagen, \\
kein Fieber die Kinder Plagen, \\
das Feuer muss von ihm weichen, \\
der bittere Hunger desgleichen, \\
das Reissen in Schulter und Knie

\initialEs dürfen Ratten und Maden, \\
seinem Rauchfleisch und Mehl nich schaden, \\
Pest, Diebe und Polizei gehen an ihm vorbei, \\
hat nie im Schuh einen Stein \\
und in Blase und Nier obendrein.

\initialD{enselben} wird Gott bekräften \\
in allen seinen Geschäften.

\vspace{1em}
\glaubenhoffenlieben

\initialW{as} steht darin? \\
Das ich Gottes Befohlener bin.

\initialG{ottes} güldener Thron \\
ist mein Bastei und Bastion. \\

\initial{J}{esu} Christ heilges Fleisch und Blut \\
ist mein Küraß und Eisenhut, \\
dass mich niemand kann fällen \\
mit Granaten oder Schrapnellen, \\
so bleib ich von Stahl und von Blei, \\
von Gift und Handschellen frei.

\initialD{er} Tod geht mir hart an die Haut, \\
aber dann ist der Weg ihm verbaut. \\
Das Feuer mag mich umlohn, \\
die bittere Flut mich bedrohn, \\
sie steigt mir doch nur bis ans Kinn, \\
weil ich Gottes Befohlener bin.

\initial{J}{edes} Elend soll mich anfassen, \\
ich bleibe ihm nicht überlassen, \\
soll leiden alle Beschwerden \\
und von keiner ertötet werden. \\
Manche Nacht  soll ich wachen u.\ bangen \\
und doch den morgen erlangen. \\
Und was meine Feinde auch hecken  \\
in ihren Kanzlein und Verstecken, \\
kein Haar auf meinem Haupt \\
ist ihnen zu krümmen erlaubt.

\initialI{ch} lebe und solcher Gestalt \\
hat die Welt an mir keine Gewalt, \\
muss alles sich schicken und fügen \\
zu Gottes und meinem Vergnügen

\vspace{1em}
\glaubenhoffenlieben

\initialEs ist nichts weiter zu schreiben. \\
Ich soll Gottes Befohlener bleiben.%
\end{document}
